\documentclass{article}
\usepackage[utf8]{inputenc}

\title{Problem 2 : Gaussian Integral (N3)}
\date{July 2019}

\begin{document}

\maketitle

\section{Interview}

\textbf{Interviewee : } Adhish Raval \hfill\break
\textbf{Interviewer: } Jil Mehta \hfill\break \break
\textbf{Why this Interviewee :} Adhish is a doctorate student doing research in the field of \underline {plasma physics}. He has been working in the field of physics since last 7 years and he has done some commendable work in his field and has also worked with some of India's well known research centres. He has also taught physics to students of Bachelor of engineering.\hfill\break\break
\textbf{Interview Questions and Responses :} \hfill\break\break
\textit{\textbf{Jil} : Hi Adhish, can you introduce yourself? \hfill\break
\textbf{Adhish} : I am Adhish Raval. I am doing my Doctorate of Philosophy in the field of Physics at the SVNIT University in Surat, India.\hfill\break
\textbf{Jil} : Great!! So Adhish, what is your research based upon?\hfill\break
\textbf{Adhish} : My research is in the field of Plasma matter. The project details are confidential and so I cannot reveal more details to you at present. \hfill\break
\textbf{Jil} : So you are doing your research at SVNIT? \hfill\break
\textbf{Adhish} : I started my research at SVNIT,Surat but as a part of my research I have worked in collaboration with many other reputed institutes. \hfill\break
\textbf{Jil} : That sounds interesting. Can you tell me which other research institutes have worked with? \hfill\break
\textbf{Adhish} : I have worked with Institute for Plasma Research, Physical Research Laboratory and with Essar company. \hfill\break
\textbf{Jil} : So during your study period, have you ever come across the concept of Gaussian Integration?\hfill\break
\textbf{Adhish} : I did study this concept during my Bachelors. But after my Bachelors, I have not worked with this concept. However, during the second year of my PhD studies, I taught a class of Bachelor students and I did teach them this concept.\hfill\break
\textbf{Jil} : Can you please elaborate as to how did you use this concept? \hfill\break 
\textbf{Adhish} : Gaussian Integral is basically used to calculate the density of various elements that are very irregular shaped.\hfill\break
\textbf{Jil} : What do you mean by irregular shapes?\hfill\break
\textbf{Adhish} : By irregular shapes I mean shapes that cannot be classified as basic shapes and are composed of various combinations of regular shapes. For eg. if you look at a funnel shaped vessel, we don't have the formula to calculate the density of funnels shaped items. You have formulas for triangles, circles, rectangles, squares, which are all regular shapes. So to calculate the density of the funnel shaped vessel, you have to divide it into various sub parts and integrate the volumes of the individual shapes and then eventually sum those individual volumes up to get the total volume of the vessel.\hfill\break
\textbf{Jil} : And are the volumes accurate?\hfill\break
\textbf{Adhish} : Not always, because while dividing irregular shapes, you cannot always separate them into exact regular shapes normally. So there are certain cumulative errors that leads to inaccurate results.\hfill\break
\textbf{Jil} : Doesn't such inaccuracies cause problems in real time when working on critical applications?\hfill\break
\textbf{Adhish} : We practically don't use this equation in real time. We have various advanced software that does this for us. For the field I work, i.e. in the field of Plasma physics,
 everything happens at atomic level.\hfill\break
\textbf{Jil} : So do you know any such field, where this formula is directly used?\hfill\break
\textbf{Adhish} : As I told you, this formula can sometimes be used in big calculations just to do small calculations. But dues to the inaccuracies in its results, it is not directly used in real time calculations. Atleast, I have not used them till date.}\hfill\break\break
\textbf{Interview Analysis: }
\begin{itemize}
    \item As per the interviewee, the Gaussian Integration method is used at the undergraduate level to make the students understand the concept of densities for various elements having irregular shapes.
    \item This would enable the students to break down the irregular shapes into smaller regular shapes and thus, help to calculate the density of the irregular shape. 
    \item However due to various cumulative errors while adding up the volumes of individual sub shapes, the actual density differs than the original one. 
    \item This may be acceptable when the fields where they are used are not critical ones. 
    \item For example calculating the volume of water that can be stored in a container. 
    \item But when the objects are to used in time critical fields like space, even a small calculation error may lead to disastrous effects.
\end{itemize}
\end{document}
